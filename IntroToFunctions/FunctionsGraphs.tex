\documentclass{ximera}
%% handout
%% space
%% newpage
%% numbers
%% nooutcomes

\usepackage{amsmath, amssymb, multicol}
\usepackage{graphicx}
\usepackage{xcolor}
%\usepackage{textcomp}
\usepackage{comment}
\usepackage{etoolbox} %for toggles
\usepackage[draft]{todonotes}
%\usepackage{tikz}


% % % % % % % % % Create instructor edition environment % % % % % % % % % % % % %
\specialcomment{instnote}{\begingroup\vskip 1em \color{blue} \noindent{\bfseries Notes: }}{\vskip 1em\endgroup}
\newcommand{\hideinstnotes}{\excludecomment{instnote}} %by including \hideinstnotes, all instructor notes will be hiden.



\def\d{\displaystyle}
\def\b{\mathbf}
\def\R{\mathbf{R}}
\def\Z{\mathbf{Z}}
\def\st{~:~}
\def\bar{\overline}
\def\inv{^{-1}}


%\pointname{pts}
%\pointsinmargin
%\marginpointname{pts}
%\addpoints
%\pagestyle{head}
%\printanswers

\hideinstnotes  %comment this line out to show instructor notes.

%\firstpageheader{Math 130}{\bf Functions and Graphs}{Week 1}
\title{Functions and Graphs}

\begin{document}

\maketitle

\noindent\textbf{Goal:} The purpose of these activities is to better understand functions and their graphs.  Here we go.
\begin{instnote}
There are actually three main goals of this first set of activities.

\begin{enumerate}
\item Remind students about the relationship between functions and graphs (and what each of those are)
\item Imporve students' ability to read and use function notation.
\item Remind students about trigonometric functions, mainly sine and cosine.
\end{enumerate}

Before beginning the activity, have a brief class discussion about what a function is.  Stress that a function is simply a relationship between two quantities with the property that each value of the first quantity is related to exactly one value of the second quantity.

\end{instnote}

\begin{problem} 
Every square has both a perimeter $p$ and an area $A$.  In other words, there is a relationship between the set of perimeters of squares and the set of areas of squares.  Is this relationship a function?
\end{problem}


%\begin{parts}
%	\part Is $A$ a function of $p$? Is $p$ a function of $A$?  Explain.
%	
%		\begin{solution}
%		Yes to both.  For each value of $p$, there is exactly one value of $A$ related to $p$.  Therefore $A$ is a function of $s$.  But also, for each value of $A$, there is exactly one value of $p$ related to $A$.  So $p$ is a function of $A$.
%		\end{solution}
%	\vfill
%	
%	\part We often use function notation such as $f(p) = A$.  What would $f(36)$ be equal to? Explain what $f(36)$ represents.  Include possible units for $36$ and $f(36)$.
%	
%		\begin{solution}
%		If $f(p) = A$, that means that the function $f$ takes a perimeter $p$ and gives an area $A$.  So $f(36) = 81$ (since the perimeter is 4 times the side length, this square has side length 9, so has area 81).  This means that a square with perimeter 36 cm has area of 81 sq cm.  Note that $f(36)$ is an area, so it has units such as square centimeters while the 36 is the perimeter, with units centimeters.
%		\end{solution}
%	
%	\vfill
%	
%	\part It is also true that every rectangle has a perimeter $p$ and an area $A$.  If we take this as our relationship, is $A$ a function of $p$?  Is $p$ a function of $A$?  Explain, using function notation (like $f(p) = A$).
%	
%		\begin{solution}
%		These are not functions.  The area is not a function of perimeter because there are multiple areas that correspond to rectangles of a given perimeter.  Say we did write $f(p) = A$.  Then $f(20)$ would be the area of a rectangle with perimeter 20 units.  If this rectangle was a square, then the side length would be 5 units, so we would have $f(20) = 25$ square units.  But the rectangle could also be a $3\times 7$ rectangle (which has perimeter $3+3+7+7 = 20$ units).  In that case we would say $f(20) = 21$ square units.  If $f$ is going to be a function, we must have $f(p)$ only ever be one value (for each specific value of $p$).  In fact there are infinitely many different rectangles with perimeter 20 but different areas.
%		
%		Similarly, the perimeter is not a function of area.  An area of 24 could correspond to a $2\times 12$ rectangle (with perimeter 28) but also a $4 \times 6$ rectangle (with perimeter 20), along with infinitely many other rectangles with different perimeters.
%		\end{solution}
%		
%	\vfill
%
%\begin{instnote}
% After students have worked in groups on the above questions, discuss solutions as a class.  Stress the notation $f(p) = A$.  Point out that $f(p)$ is a number, not an instruction to compute a number.
%\end{instnote}	
%\end{parts}
%\newpage
%
%\begin{instnote}
%	The goal of the next question is to remind students about graphing and to illustrate the relationship between the graph of an equation and the graph of a function.  The important take-away is that when we graph a function $f$, we are graphing $y = f(x)$.  Point out that $f(x)$ \emph{is} the $y$-value corresponding to the $x$-value.
%\end{instnote}
%
%\question A nice way to represent some functions is with a \emph{graph}.  Before we think about how to do this, let's review what it means to graph an \emph{equation}.
%
%\begin{parts}
%	\part Consider the equation $y = 2x - 1$.  We can pick pairs $(x,y)$ and plug them into the equation.  Sometimes this will give a true equation, for example the pair $(3,5)$.  Some pairs will give a false equation, such as $(7,1)$.  Draw an $xy$-plane below and plot all pairs that make the equation true.
%	
%	\begin{solution}
%	The pairs that make the equation true will form a line with slope 2 and $y$-intercept 1.
%	\end{solution}
%	
%	\vfill
%	
%	\part Repeat the previous question, this time using the equation $x^2 + y^2 = 25$.  Some points to try: $(-4,3)$, $(6,1)$, etc.  What is the largest and smallest each of $x$ and $y$ could be?
%	
%	\begin{solution}
%	The graph will be a circle of radius 5, centered at the origin.
%	\end{solution}
%	
%	\vfill
%	
%	\part Both of the equations above represent relationships between values of $x$ and values of $y$.  Are the relationships functions?  How could you use function notation to represent the relationship?
%	
%	\begin{solution}
%	The first equation does represent a function.  We can say that $y$ is a function of $x$, and write $f(x) = y$.  In other words, to get the $y$ value for a given $x$ value, we compute $f(x) = 2x - 1$, which gives $y$ since $y = 2x-1$.
%	
%	The second equation is NOT a function.  For example, when $x = 3$, we could have $y$ be 4 or $-4$ and still make the equation true.
%	\end{solution}
%\end{parts}
%	
%
%\newpage
%
%\question 



\end{document}


