\documentclass{ximera}
%% handout
%% space
%% newpage
%% numbers
%% nooutcomes

\usepackage{amsmath, amssymb, multicol}
\usepackage{graphicx}
\usepackage{xcolor}
%\usepackage{textcomp}
\usepackage{comment}
\usepackage{etoolbox} %for toggles
\usepackage[draft]{todonotes}
%\usepackage{tikz}
\pgfplotsset{compat=1.7}

% % % % % % % % % Create instructor edition environment % % % % % % % % % % % % %
\specialcomment{instnote}{\begingroup\vskip 1em \color{blue} \noindent{\bfseries Notes: }}{\vskip 1em\endgroup}
\newcommand{\hideinstnotes}{\excludecomment{instnote}} %by including \hideinstnotes, all instructor notes will be hiden.


\def\d{\displaystyle}
\def\b{\mathbf}
\def\R{\mathbf{R}}
\def\Z{\mathbf{Z}}
\def\st{~:~}
\def\bar{\overline}
\def\inv{^{-1}}


%\pointname{pts}
%\pointsinmargin
%\marginpointname{pts}
%\addpoints
%\pagestyle{head}
%\printanswers

\hideinstnotes  %comment this line out to show instructor notes.

%\firstpageheader{Math 130}{\bf Functions and Graphs}{Week 1}
\title{Functions and Graphs}

\begin{document}
\begin{abstract}
The purpose of these activities is to better understand functions and their graphs.  But really to practice with Ximera.
\end{abstract}
\maketitle



\begin{instnote}
There are actually three main goals of this first set of activities.

\begin{enumerate}
\item Remind students about the relationship between functions and graphs (and what each of those are)
\item Imporve students' ability to read and use function notation.
\item Remind students about trigonometric functions, mainly sine and cosine.
\end{enumerate}

Before beginning the activity, have a brief class discussion about what a function is.  Stress that a function is simply a relationship between two quantities with the property that each value of the first quantity is related to exactly one value of the second quantity.

\end{instnote}


Every square has both a perimeter $p$ and an area $A$.  In other words, there is a relationship between the set of perimeters of squares and the set of areas of squares.  Is this relationship a function?

\begin{exercise}
If $p = 20$, then $A = \answer{25}$.
\begin{hint}
What is the side length of a square with this perimeter?
\end{hint}
\end{exercise}

\begin{exercise}
If $A = 36$ then $p = \answer{24}$.
\begin{hint}
What is the side length?
\end{hint}

\begin{solution}
If the area is 36, then the side length must be 6, since $6\times 6 = 36$.  The perimeter is 4 times the side length, so the perimeter is 24.
\end{solution}
\end{exercise}

\begin{question}

Every square is also a \wordChoice[circle, dog, triangle]{rectangle}

\end{question}




\end{document}


