\documentclass{ximera}
%% handout
%% space
%% newpage
%% numbers
%% nooutcomes


\usepackage{amsmath, amssymb, multicol}
\usepackage{graphicx}
\usepackage{xcolor}
%\usepackage{textcomp}
\usepackage{comment}
\usepackage{etoolbox} %for toggles
\usepackage[draft]{todonotes}
%\usepackage{tikz}
\pgfplotsset{compat=1.7}

% % % % % % % % % Create instructor edition environment % % % % % % % % % % % % %
\specialcomment{instnote}{\begingroup\vskip 1em \color{blue} \noindent{\bfseries Notes: }}{\vskip 1em\endgroup}
\newcommand{\hideinstnotes}{\excludecomment{instnote}} %by including \hideinstnotes, all instructor notes will be hiden.


\def\d{\displaystyle}
\def\b{\mathbf}
\def\R{\mathbf{R}}
\def\Z{\mathbf{Z}}
\def\st{~:~}
\def\bar{\overline}
\def\inv{^{-1}}


%\pointname{pts}
%\pointsinmargin
%\marginpointname{pts}
%\addpoints
%\pagestyle{head}
%\printanswers

\hideinstnotes  %comment this line out to show instructor notes.

%\firstpageheader{Math 130}{\bf Functions and Graphs}{Week 1}


\begin{document}

\title{Factoring}

\begin{abstract}
Our next goal is to recall how to factor.  What does this even mean?
\end{abstract}
\maketitle



When we factor, we break up an expression into its parts, such that when you \emph{multiply} all the parts together, you get the expression.    That is, we are decomposing the expression into its ``factors'' -- parts with respect to multiplication.


\begin{theorem}
Factoring is fun.
\end{theorem}

\begin{problem}
This is a question.  The answer is: $\answer{42}$
\end{problem}

\begin{example}
Here is an example of a list.
\begin{itemize}
\item One item.
\item Another.
\item This is not an item.  Okay, it really is.
\end{itemize}
\end{example}

This is the end of the file.
\end{document}


