\documentclass{ximera}
%% handout
%% space
%% newpage
%% numbers
%% nooutcomes

%\usepackage{amsmath, amssymb, multicol}
%\usepackage{graphicx}
%\usepackage{xcolor}
%%\usepackage{textcomp}
%\usepackage{comment}
%\usepackage{etoolbox} %for toggles
%\usepackage[draft]{todonotes}
%%\usepackage{tikz}
%\pgfplotsset{compat=1.7}

%% % % % % % % % % Create instructor edition environment % % % % % % % % % % % % %
%\specialcomment{instnote}{\begingroup\vskip 1em \color{blue} \noindent{\bfseries Notes: }}{\vskip 1em\endgroup}
%\newcommand{\hideinstnotes}{\excludecomment{instnote}} %by including \hideinstnotes, all instructor notes will be hiden.


\def\d{\displaystyle}
\def\b{\mathbf}
\def\R{\mathbf{R}}
\def\Z{\mathbf{Z}}
\def\st{~:~}
\def\bar{\overline}
\def\inv{^{-1}}


%\pointname{pts}
%\pointsinmargin
%\marginpointname{pts}
%\addpoints
%\pagestyle{head}
%\printanswers

%\hideinstnotes  %comment this line out to show instructor notes.

%\firstpageheader{Math 130}{\bf Functions and Graphs}{Week 1}

\title{Factoring Numbers}

\outcome{Let's see how to factor numbers}

\begin{document}



\begin{abstract}
Before we can factor polynomials, let's review how we factor numbers.
\end{abstract}
\maketitle


What does it mean to \emph{factor}?  To get a feel for this, we will try some simple number factoring problems.

\begin{problem}
What is the smallest factor of 45?  $\answer{3}$
\end{problem}


Which is not a factor of 54?  
\begin{multipleChoice}
\choice{3}
\choice[correct]{5}
\choice{9}
\end{multipleChoice}



\end{document}


